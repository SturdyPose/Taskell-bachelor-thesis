\documentclass[male,czech]{kithesis}

\usepackage{graphics}
\usepackage{array}

% zde nastavte základní informace o bakalářské práci
\newcommand{\AUTOR}{Oleg Musijenko}
\newcommand{\TITULcz}{Tacit programming - návrh doménově specifického jazyka a implementace jeho interpretu} % titul v českém jazyce
\newcommand{\TITULen}{Tacit programming - design of a domain specific language and implementation of it's interpreter} % titul v anglickém jazyce
\newcommand{\KLICOVASLOVAcz}{bakalářská práce, odborný text, programování} % klíčová slova v českém jazyce
\newcommand{\KLICOVASLOVAen}{bachelor thesis} % klíčová slova v anglickém jazyce
\newcommand{\VEDOUCI}{Mgr. Jiří Fišer, Ph.D.}    

\newcommand{\PROGRAM}{Aplikovaná informatika}    
\newcommand{\OBOR}{Informační systémy}    

% nastavení fontů (liší se podle TeXovského stroje)
\iftutex
\usepackage{fontspec}
\setmainfont{Libertinus Serif} % použito je opensource písmo Libertinus, lze použít jakékoliv jiné rozumné
\setsansfont{Libertinus Sans}
\setmonofont[Scale = MatchLowercase]{Libertinus Mono}
\usepackage{unicode-math}  
% písmo pro matematiku, vhodná písma viz  	např. https://developer.mozilla.org/en-US/docs/Mozilla/MathML_Project/Fonts
\setmathfont{Libertinus Math}
\else
\usepackage[utf8]{inputenc}
\usepackage[T1]{fontenc}
\usepackage{libertinus}
\usepackage{amsmath,amssymb}
\usepackage{libertinust1math}
\fi

\usepackage[style=authoryear,backend=biber]{biblatex}
\usepackage{minted}

\addbibresource{bibliografie.bib}
% vylepšení vzhledu
\usepackage{microtype}
%\sloppy %odpoznámkujte pokud vám často přetékají řádky
\renewcommand{\arraystretch}{1.23} % vertikální roztažení tabulek o 23%
            
% nastavení pro sazbu výpisu kódu
\usepackage{listings}

\lstset{ %
  language=Haskell,                % hlavní programovací jazyk, seznam podporovaných viz https://en.wikibooks.org/wiki/LaTeX/Source_Code_Listings
  basicstyle=\ttfamily,    
  showspaces=false,                % show spaces adding particular underscores
  showstringspaces=true,           % underline spaces within strings
  showtabs=false,                  % show tabs within strings adding particular underscores
  frame=single,                    % adds a frame around the code
  tabsize=3,                       % sets default tabsize to 3 spaces
  breaklines=true,                 % sets automatic line breaking
  breakatwhitespace=false,         % sets if automatic breaks should only happen at whitespace
  keywordstyle=\bfseries,          % keyword style
  commentstyle=\rmfamily,          % comment style
  stringstyle=\itshape,            % string literal style
}

% nastavení hypertextových odkazů a PDF metainformací
\usepackage{url} % přidává příkaz url pro sazbu url
\usepackage[unicode=true,colorlinks=true,
            pdftitle={\TITULcz},pdfauthor={\AUTOR},
            pdfkeywords={\KLICOVASLOVAcz}]{hyperref}
            
\pagestyle{fancy} % aktivování stylu záhlaví a zápatí z kithesis

%---------------------------------------------------------

\begin{document}
\thispagestyle{empty}
\begin{center}
{\Huge Univerzita Jana Evangelisty Purkyně \\
v~Ústí nad Labem}
\\[16pt]
{\huge Přírodovědecká fakulta}

\vspace{2cm}
\resizebox{8cm}{!}{\includegraphics{LOGO_PRF_CZ_RGB_standard.jpg}}

\vspace{2cm}
{
\huge
\TITULcz\par

\vspace{0.5em}
\LARGE\scshape bakalářská práce
}
\end{center} 
 
\vfill
{
\large
\begin{tabular}{>{\bfseries}rl}
    Vypracoval: 	& \AUTOR\\
    Vedoucí práce: 	& \VEDOUCI\\
&\\
Studijní program:       & \PROGRAM\\
Studijní obor:          & \OBOR\\
\end{tabular} 
}
\vspace{1.5cm}
\begin{center}
\Large\scshape   Ústí nad Labem \the\year
\end{center}

\cleardoublepage
\thispagestyle{empty}

\textbf{Zde bude vloženo zadání bakalářské práce!!}

Zadání bakalářské je možné získat až
v okamžiku, kdy je práce schválena ve STAGu vedoucím práce, vedoucím katedry a garantem oboru a nelze tak učinit elektronicky.

Požádejte o něj vedoucího katedry informatiky. Zadání vám bude doporučeno v podobě elektronicky podepsaného PDF, které vložíte na místo toho listu
(dvojstránky) některým z nástrojů pro práci s PDF dokumenty.

\cleardoublepage
\thispagestyle{empty}

\textbf{Prohlášení}

Prohlašuji, že jsem tuto bakalářskou práci vypracoval\ifthenelse{\boolean{feminum}}{a}{} samostatně a použil\ifthenelse{\boolean{feminum}}{a}{}
jen pramenů, které cituji a uvádím v přiloženém seznamu literatury.

\vspace{1em}
Byl\ifthenelse{\boolean{feminum}}{a}{} jsem seznámen\ifthenelse{\boolean{feminum}}{a}{} s tím, že se na moji práci vztahují práva a povinnosti vyplývající ze zákona č. 121/2000 Sb., ve znění zákona č. 81/2005 Sb., autorský zákon, zejména se skutečností, že Univerzita Jana Evangelisty Purkyně v Ústí nad Labem má právo na uzavření licenční smlouvy o užití této práce jako školního díla podle § 60 odst. 1 autorského zákona,s tím, že pokud dojde k užití této práce mnou nebo bude poskytnuta licence o užití jinému
subjektu, je Univerzita Jana Evangelisty Purkyně v Ústí nad Labem oprávněna ode mne požadovat přiměřený příspěvek na úhradu nákladů, které na vytvoření díla vynaložila, a to podle okolností až do jejich skutečné výše.

\vspace{1em}
V Ústí nad Labem dne \today \hspace{0.3\textwidth} Podpis:


\cleardoublepage
\thispagestyle{empty}
~\vfill

\begin{flushright}
  Děkuji vedoucímu práce Mgr. Jiřímu Fišerovi, Ph.D.\\ za neocenitelné rady a pomoc při tvorbě bakalářské práce.
\end{flushright}

\cleardoublepage
\thispagestyle{empty}

\textbf{\textsf{Abstrakt}}

\textsc{\TITULcz}

Abstrakt shrnuje základní motivaci práce (kontext), hlavní cíl a následně jednotlivé
autorské kroky k~jeho splnění (co bylo uděláno od úvodních rešerší, přes návrh, implementaci k případnému nasazení. Minimální rozsah je 800 znaků (maximální půl strany).

\textbf{\textsf{Klíčová slova}}

seznam klíčových slov (obecných termínů vystihujících téma práce) v počtu dva až deset 

\vspace{1em}
\hrulefill
\vspace{1em}

\textbf{\textsf{Abstract}}

\textsc{\TITULen}

Translation of Czech abstract.

\textbf{\textsf{Key words}}

Translation of czech key words.

\tableofcontents

\addchap{Úvod}


\chapter{Tacit programming}

\textbf{Tacit programming} je styl programování, kde nevyužíváme parametry funkcí, 
ale funkce řetězíme, nebo kompozicujeme. Ukažme si jednoduchý a intuitivní příklad v javascriptu.

\begin{minted}{js}
  fetch("APIURL")
  .then(x => fancyFunction(x))
  .then(x => console.log(x))
  .catch(e => throw Error(e))
\end{minted}

Po získání dat provedeme transformaci dat (definice fancyFunction není v tomto kontextu důležitá, stačí pouze vědět, že vrací Promise/Slib) a poté zapíšeme výsledek do Konzole. 
Toto je jeden ze způsobů, jak můžeme na sebe řetězit funkce zpětného volání ("Callbacks").

Toto je zcela běžná praxe javascript programátorů, ale bohužel má jednu malou nevýhodu.
Tvoří se zde zbytečná anonymní funkce ("arrow function nebo-li šipková") a pokud bychom prohlubovali čím dál víc 
zásobník volání, mohou nám tyto anonymní funkce zabírat paměť. 

\begin{minted}{js}
  fetch("APIURL")
  .then(fancyFunction)
  .then(console.log)
  .catch(throw Error)
\end{minted}

Zde se nachází kód, který dělá stejné instrukce, jako ten předchozí. Rozdíl je ten, že je zapsán jako \textit{beztečkový "point free"}.
Takto programátor předchází potřebě dělat wrappovací funkce.

Na následujícím přikladě si ukážeme, jak funguje \textbf{currying} a proč souvisí s tacit programováním.

\begin{minted}{js}
const curry = (f) => a => b => f(a,b);

const sayHello = (a, b) = `Hello ${a} from ${b}`;

const applyToFunctionArray = (input,...args) => args.map(a => a(input))

const partiallyAppliedData = ["A", "B", "C"].map(curry(sayHello)); 
// [(b) => "Hello A from ${b}", 
//  (b) => "Hello B from ${b}", 
//  (b) => "Hello C from ${b}"]

const partiallyAppliedData2 = ["A", "B", "C"].map(curry(sayHello)(1)); 
// ["Hello A from 1", 
//  "Hello B from 1", 
//  "Hello C from 1"]
\end{minted}

Curry funkce zařizuje, že máme pro každý argument vlastní funkci. V čem je toto výhodné?
Například je zde uvedené pole, které se skládá z částečně aplikovaných funkcí. 
Takto může programátor u předchozí ukázky naiterovat odpověď ze serveru do objektu, které je závislé na třeba na uživatelském vstupu. 

Zajímavější část je u \textit{partiallyAppliedData2}.

\section{Principy a odlišnosti od klasického procedurálního paradigmatu}

\section{Rešerše existujících implementací}

\chapter{DSL - principy a využití}

\chapter{Návrh vlastního DSL}

\chapter{Implementace interpretu navrženého DSL}

\chapter{Ověření použitelnosti (testování funkčnosti, praktické příklady využití)}

\chapter{Závěr}

\chapter{Citace}


\cite{Katuscakc}
\cite{IntroToLLVM}
\printbibliography


\appendix




\end{document}